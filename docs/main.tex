\documentclass[12pt, a4paper]{article}

\usepackage[spanish]{babel}

\usepackage{amsmath} % Escritura mejorada de fórmulas matemáticas
\usepackage{graphicx} % Inserción de gráficos
\usepackage{algorithm}
\usepackage{algpseudocode}

\usepackage[colorlinks=true, citecolor=blue]{hyperref}


\title{Crossword CBD}
\author{Pedro González Marcos}
\date{27 de Mayo 2024}

\begin{document}

\maketitle

\tableofcontents

\section{Introducción}

Esto es un texto de ejemplo \cite{rahman2016feature} \cite{foo}



\section{Cuestiones previas}

\subsection{Convención de crucigrama americano}

Para la implementación del algoritmo que encuentra las palabras dentro del
crucigrama, se han hecho las siguientes suposiciones:

\begin{itemize}
	\item Las palabras están formadas por al menos 3 letras.
	\item El número de palabras que hay en un crucigrama es la suma de
	las palabras en horizontal y en vertical.
\end{itemize}

\section{Modelo de datos}

Para lsdfsdf

\subsection{Algoritmo}



$$
\left( 2+2 \right) 
$$

\begin{algorithm}
\caption{Mi primera chamba}
\begin{algorithmic}
\Require $n \geq 0 \vee x \neq 0$
\Ensure $y = x^n$
	\While{$N \neq 0$}
	\State $X$
\EndWhile
\end{algorithmic}
\end{algorithm}

\section{Conclusiones}

\bibliography{biblio}
\bibliographystyle{alpha}


\end{document}